% \newpage para arrancar una pagina nueva
% \mbox y \ fbox para evitar que te corte al medio una palabra que vos querer tenr toda junta
% \mbox{} se puede usar cuando 2 letras se pegan. entonces lo ponemos entre medio de esas 2 letras
% The \part command does not influence the numbering sequence of chapters.
% \footnote{footnote text} para poner notas al pie de la pagina
% \underline{text}
% \emph{text} pone en italica
% \begin{enumerate} o {itemize} y despues listas con \item o {description} y listas con \item[palabra]



\documentclass[a4paper,10pt, notitlepage]{article}
\usepackage[utf8x]{inputenc}
\usepackage[english,spanish]{babel}

\usepackage{graphicx}

\pagestyle{plain} %o pagestyle{plain} , valor defecto   {headings}
% define the title
\title{\textbf{TP0 Organización de las Computadoras (66.20)}}
% \author{Rodriguez Genaro, Leandro \and Padron: 92098 \\ 
\author{}
\date{}


\begin{document}
% generates the title
\maketitle

\begin{center}
Grupo Nro. X - 1er. Cuatrimestre de 2012                  \\
66.20 Organización de Computadoras                        \\
Facultad de Ingeniería, Universidad de Buenos Aires       \\
\end{center}

% Inclusión de una imagen en formato EPS (Encapsulated Postscript).
\begin{figure}[!htp]
\begin{center}
\includegraphics[width=0.5\textwidth]{logo_fiuba.jpg}
\end{center}
\end{figure}

\begin{flushleft}
{\renewcommand{\arraystretch}{2.5}
\renewcommand{\tabcolsep}{1.2cm}
\begin{tabular}{ l l }
  Rodriguez Genaro, Leandro & \textit{Padrón Nro. 92.098} \\
  \texttt{leandrorodriguezg@yahoo.com.ar} \\
  \hline
  Reale, Tomás & \textit{Padrón Nro. 92.255} \\
  \texttt{tomasreale@gmail.com} \\
  \hline
  Piechotka, Federico & \textit{Padrón Nro. 92.216} \\
  \texttt{$error si pones el guion bajo de f piecho@hotmail.com$} \\
  \hline
\end{tabular}}
\end{flushleft}

%aca vendria a termina la caratulas
%ver si se puede evitar que cuente en la numeracion

% Quita el número en la primer página.
\thispagestyle{empty}

\newpage

%resumen
\begin{abstract}
Vendria a ser el resumen. Vamos a poner un resumen? (parece que es necesario y de menos de 140 palabras)
\end{abstract}

%inserta el indice
\tableofcontents
\newpage

\section{\underline{Introducción}}
Ponele que aca va una introducción.
\section{Desarrollo}
\ldots{} o algo por el estilo.

\section{\emph{Probando el modo ecuaciones}}

\subsection{Ecuación 1}
% Example 1
\ldots when Einstein introduced his formula
\begin{equation}
e = m \cdot c^2 \; ,
\end{equation}
which is at the same time the most widely known
and the least well understood physical formula.

\subsection{Ecuación 2}
% Example 2
\ldots from which follows Kirchhoff’s current law:
\begin{equation}
\sum_{k=1}^{n} I_k = 0 \; .
\end{equation}
Kirchhoff’s voltage law can be derived \ldots

\subsection{Ecuación 3}
% Example 3
\ldots which has several advantages.
\begin{equation}
I_D = I_F - I_R
\end{equation}
is the core of a very different transistor model. \ldots


\section{Conclusiones}

Todo el chamuyo a mandar aca. Cosas como que el merge sort es altamente superior para x casos. Que con archivos enormes paso z cosa.
Tambien se puede poner algo de latex. Nose, usemos la imaginacion


% Citas bibliogr�ficas.
\begin{thebibliography}{99}

\bibitem{INT06} Intel Technology \& Research, ``Hyper-Threading Technology,'' 2006, http://www.intel.com/technology/hyperthread/.

\bibitem{HEN00} J. L. Hennessy and D. A. Patterson, ``Computer Architecture. A Quantitative
Approach,'' 3ra Edición, Morgan Kaufmann Publishers, 2000.

\bibitem{LAR92} J. Larus and T. Ball, ``Rewriting Executable Files to Mesure Program Behavior,'' Tech. Report 1083, Univ. of Wisconsin, 1992.

\end{thebibliography}

\end{document}
